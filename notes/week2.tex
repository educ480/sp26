\documentclass[11pt]{article}
\usepackage[utf8]{inputenc}
\usepackage[T1]{fontenc}
\usepackage{geometry}
\geometry{margin=1in}
\usepackage{enumitem}
\usepackage{hyperref}
\usepackage{xcolor}
\usepackage{fancyhdr}
\usepackage{sectsty}
\sectionfont{\large\bfseries}
\subsectionfont{\normalsize\bfseries}

\begin{document}
\begin{center}
\textbf{Reading Notes: Large ``If Justice Is Our Objective'': Diaspora Literacy, Heritage Knowledge, and the Praxis of Critical Studyin' for Human Freedom  \\ Joyce E. King (2006)}\\
%\textbf{Study Guide for Students and Educators}
\end{center}

\section{Overview and Thesis}
Joyce E. King argues that true educational justice requires moving beyond mere access to a flawed, ideologically racist curriculum. Instead, educators must embrace \textbf{Critical Studyin'}, a praxis rooted in Black Studies that uses \textbf{Diaspora Literacy} and \textbf{Heritage Knowledge} to liberate learners from dehumanizing myths and foster human freedom.

\begin{itemize}
\item \textbf{Key Problem}: Post-Brown v. Board education perpetuates the ``national mythology of Black inferiority'' through distorted knowledge about Africa, slavery, and Black culture.
\item \textbf{Solution}: Recover ``stolen knowledge, labor, and culture'' via culturally grounded pedagogy.
\item \textbf{Moral Imperative}: Educators must counter ``dysconscious racism'' to realize Brown's unfulfilled promise.
\end{itemize}

\section{Key Concepts}
\subsection{Diaspora Literacy}
Ability to ``read the word and the world'' from an informed, indigenous African/diasporic perspective. Decodes alienating narratives in everyday Black life.

\subsection{Heritage Knowledge}
A people's collective memory and historical consciousness---a ``cultural birthright.'' Essential for identity, pride, and liberation (``like a child to her mother'').

\subsection{Critical Studyin'}
Praxis inspired by enslaved Africans ``studyin' freedom.'' Morally engaged teaching that frees cognition and emotion from ideological constraints. ``Practice-to-theory'': theorizes Black experience to reconnect dismembered realities.

\subsection{Alterity and Perspective Advantage}
Black people's ``second sight'' (Du Bois) or ``perspective advantage'' (Wynter) from living as the dehumanized ``Other''---enables critique of white supremacy's costs to all humanity.

\section{Examples of Ideological Distortions}
\begin{itemize}
\item \textbf{Textbooks}: Africans ``sold their own people'' into slavery (ignores lineage, European wars); Cro-Magnon ``looked just like us'' vs. African origins.
\item \textbf{Slavery Narratives}: Ignores trauma, portrays as ``salvation'' or benign.
\item \textbf{Conceptual Blackness}: ``NHI'' (No Humans Involved); biological inferiority myths (The Bell Curve).
\end{itemize}

\section{Critical Studyin' Examples}
\begin{enumerate}
\item \textbf{Songhoy-senni ``barnya'' (slave)}: ``Someone who doesn't even have a mother''---highlights lineage protection absent in chattel slavery.
\item \textbf{Aunt Jemima}: Counternarrative recovers agency of enslaved women who preserved rice knowledge (Carolina Gold).
\item \textbf{Stolen Culture}: Black music (spirituals, blues) appropriated (Elvis, Eminem); Douglass/Brown as moral symbiosis.
\item \textbf{Boondocks}: Satirizes power via Black cultural signs (Ebonics, signifying).
\end{enumerate}

\section{Discussion Questions}
\begin{enumerate}[leftmargin=*]
\item How does ``dysconscious racism'' limit teachers' moral agency? Examples?
\item Compare Diaspora Literacy to ``adding multicultural content''---what's missing in the latter?
\item Why does King say miseducation harms white students too? Evidence?
\item Analyze one example (e.g., Songhoy-senni): How does it rewrite knowledge?
\item Position: Are educators obligated to teach heritage-centered praxis? Defend with text.
\end{enumerate}

\section{Position Statement Template}
\begin{quote}
If justice is our objective, educators \underline{must/need not} engage in Critical Studyin' because \underline{[thesis with evidence]}.

Challenge: \underline{[counterargument]}. Response: \underline{[King's framework]}
\end{quote}

\section{Terms to Know}
\begin{description}
\item[Dysconsciousness] Impaired thinking accepting dominance as normal.
\item[Nihilated Identity] Total denial of humanity via racial subordination.
\item[Will to Blackness] Chosen cultural identification resisting nihilism.
\end{description}

\section{Further Reading}
\begin{itemize}
\item Woodson, \textit{The Miseducation of the Negro} (1933)
\item Wynter on ``genre of the human''
\item Freire, \textit{Pedagogy of the Oppressed}
\end{itemize}

\end{document}
